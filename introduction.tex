\vspace{-10pt}
\section{Introduction}
\label{sec:intro}
\vspace{-5pt}
Multi-streamed SSDs provide a special mechanism,
called streams, for a host system to prevent data with different lifetimes 
from being mixed into the same block~\cite{T10, MultiStream}.
When the host system maps two data $D_1$ and $D_2$ to 
different streams $S_1$ and $S_2$, a multi-streamed SSD guarantees that 
$D_1$ and $D_2$ are placed to different blocks.   
Since streams, when properly managed, can be very effective in minimizing 
the copy cost of garbage collection (GC), they
can significantly improve both the performance and lifetime of 
flash-based SSDs~\cite{MultiStream, Level, FStream, AutoStream}.

In order to achieve high performance on multi-streamed SSDs, data with similar 
{\it future} update times~\cite{PCHa}
should be allocated 
to the same stream, so that the copy cost of GC can be minimized.
However, since it is difficult to know the future update times {\it a priori} when they are written,
stream allocation decisions are often {\it manually} made 
by programmers based on their expertise (or experience) 
on the application~\cite{MultiStream, Level} or the file system~\cite{FStream}.  
Furthermore, these manual techniques assume 
that the number of streams in an SSD is not changing, 
thus requiring manual modifications whenever the number of streams in the SSD changes.
In this paper, our goal is to develop 
a {\it fully automatic} technique for managing streams 
which is applicable for {\it any} multi-streamed SSD.

To the best of our knowledge, AutoStream~\cite{AutoStream} is the only automatic 
stream management technique
without additional manual work.  
However, since AutoStream predicts data lifetimes using the update frequency 
of the logical block address (LBA), it does not work well with modern append-only workloads 
such as RocksDB~\cite{RocksDB} or Cassandra~\cite{Cassandra}.  
Unlike conventional update workloads where data written to the same LBAs 
often show strong update locality, 
append-only workloads make it impossible to predict data lifetimes 
from LBA characteristics (such as access frequency or access patterns).  
For example, as shown in Fig.~\ref{fig:lba_lifetime}(b), 
data written to a fixed LBA range over times in RocksDB 
show widely varying data lifetimes, 
thus making it difficult to allocate streams based on LBA characteristics.

In this paper, we propose a fully automatic stream management technique, called {\sf PCStream}, 
for multi-streamed SSDs based on program contexts (PCs).
Since the key motivation behind {\sf PCStream} was 
that data lifetimes should be estimated at a higher abstraction level than LBAs, 
{\sf PCStream} employed a write program context\footnote{Since we are interested in write-related 
system calls such as write() and writev() in the Linux kernel, 
we call their related program contexts as 
{\it write program contexts.} In the rest of the paper, we use 
{\it program contexts} interchangeably with {\it write program contexts}.}  
as a stream management unit.
A program context~\cite{PC, PC2}, which represents a particular execution phase of a program, 
is known to be an effective hint in separating data with different lifetimes~\cite{PCHa}.  
{\sf PCStream} automatically maps an identified program context to a stream.  
Since program contexts can be computed during run time, 
{\sf PCStream} does not need any manual work.   
%In order to handle append-only workloads, 
%{\sf PCStream} extended the definition of the data lifetime 
%so that the effect of the TRIM command~\cite{TRIM} can be accounted for. 

Although most program contexts show that their data lifetimes are 
distributed with small variances, we observed a few outliers 
whose data lifetimes have rather large variances.
In {\sf PCStream}, 
when such a PC {\it pID} is observed (which was mapped to a stream {\it sID}), 
the long-lived data of {\it pID} are moved to the substream of {\it sID}
during GC.  
The substream prevents the long-lived data of the stream {\it sID} 
from being mixed with future short-lived data of the stream {\it sID}.

In order to evaluate the effectiveness of {\sf PCStream}, 
we have implemented {\sf PCStream}
in the Linux kernel (ver. 4.5) and measured write amplification factor (WAF) values 
using RocksDB on a Samsung PM963 SSD 
as well as an SSD emulator.
Our experimental results show that {\sf PCStream}
can reduce the GC overhead as much as a highly-optimized 
manual stream management technique while requiring no code modification.  
Furthermore, PCStream outperformed AutoStream by reducing the average WAF by 30\%.

The rest of this paper is organized as follows. 
We explain the key motivations behind {\sf PCStream} in Section 2. 
Section 3 describes 
the design of {\sf PCStream}.
The experimental results are shown in Section 4. 
Finally, we conclude in Section 5 with a summary and future work. 

