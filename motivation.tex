\section{Motivation}
Recently, various studies are proposed to exploit the stream feature.
First, Kang et al.~\cite{MultiStream} proposed that the application
is modified to manually assign streams.
Since an application knows the lifetime of the data best, this approach
is very effective in reducing WAF.
However, in order to properly specify streams in the application, the programmer must
fully understand the lifetime characteristics of the data.
Also when multiple applications try to assign streams, a centralized stream assignment
is required to avoid conflicts.
Second, FStream~\cite{FStream} separates short-lived data, e.g., file system metadata and
journal, using the file system information. 
FStream does not require a burden on the programmer, but the system developer is still burdened
to identify short-lived data of the application, e.g., log data of key-value store, based on the file extension.
In addition, those manual techniques are unable to adapt the stream mapping when the workload or application changes.
These limitations of the manual approach can be overcome by the automatic approach.
Lastly, unlike other schemes, AutoStream~\cite{AutoStream} is aimed to automate the process of mapping 
write I/O operations to an SSD stream.
However, since AutoStream relies on the past LBA access patterns, it is not practical when the data are written in
append-only manner, as modern key-value store.

\begin{figure}[!b]
	\vspace{-15pt}
	\centering
	\includegraphics[width=0.9\linewidth]{figure/lba_lifetime} 
	\vspace{-10pt}
	\caption{The lifetime distribution according to the LBA.}
	\label{fig:lba_lifetime}
\end{figure}

In order to see the relationship between the address and the lifetime of data of the append-only workload,
we analyzed the write pattern of RocksDB~\cite{RocksDB}, one of the popular key-value store.
Figure~\ref{fig:lba_lifetime} shows the lifetime of the data written to the entire logical address space.
The x-axis represents the logical block address and 
and y-axis represents the lifetime of data.
As shown in the Figure~\ref{fig:lba_lifetime}, all the data have very different lifetimes regardless of their 
address.
It means that the address is not enough to properly distinguish the lifetime of data.
For the automatic stream allocation, thus, more information of the application layer is required.

%The address-based approach can not properly distinguishes the lifetime of the data written by
%the append-only workload.

To sum up, although the application knows the lifetime of the written data, it is hard to 
deliver the information to the system layer without modification.
In this paper, we propose PCStream, which is an automatic stream allocation technique with
more accurate lifetime prediction by obtaining the context informataion of an application.

