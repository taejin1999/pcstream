\section{Conclusion}
In this paper, we presented a new fully automatic stream management technique,
called {\sf PCStream}, which can improve the accuracy of data separation
using the application level information.
%In the modern append-only workload, since the hotness of data is regardless of 
%the its address, 
%LBA-based existing automatic stream management technique have limitations on 
%identifying the hotness of data.
{\sf PCStream} was motivated by our observation that data lifetimes can be
reliably predicted using write program contexts which is from the higher
abstraction level than LBAs, which was used for the existing automatic technique.
By allocating the identified PCs into streams according to the mapping policy,
{\sf PCStream} can separate different lifetime data. 
For some PCs showing various lifetime, {\sf PCStream} moves the long-lived data of 
current stream to its substream during the garbage collection.
Experimental result showed that {\sf PCStream} reduces WAF by 40\% over the existing
automatic technique.

As future work, we will develop the modules of {\sf PCStream} in various directions. 
In particular, a PC clustering module will be developed to find and group the PCs with similar lifetimes
in case of the number of PCs is larger than the number of streams.
We will also develop a stream allocator and mapping policy that dynamically allocates 
streams by monitoring PC lifetime information for 
applications with complex write patterns or multiple instances.
