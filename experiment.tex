\section{Experimental Results}
For our evaluation, we implemented PC extractor module at the 
system call layer 
and Stream Allocation module at the block I/O layer in the Linux kernel 3.16.

In order to evaluate the effectiveness of {\sf PCStream},
we implemented the multi-stream feature and substream concept
to the in-house SSD emulator
based on the open flash development platform~\cite{AMF}.
The number of streams was set to 8 for the evaluation.
The SSD emulator was 8GB with four channels with four ways, and 
the number of blocks per a parallel unit was 512 and
the number of pages per block was 256 with 4KB-sized page.
Due to the limited capacity of the emulator, 
we scaled down RocksDB configuration.
The base file size is set to 8 MB
with the size of key-value is 8 KB and the number of levels was set to 4.
However, the size of level multiplier is remained to be 10 as an usual setting,
which means the size of the next level is 10 times larger than previous level,
to maintain the level access patterns during the compaction.

For benchmarks, we used three scenarios of db\_bench and db\_stress of RocksDB.
The update random scenario (read-modify-write for random keys), {\tt UR}, and 
the append random scenario (read-modify-write with growing values), {\tt AR}, are
for intensively updating key-value pairs.
The fill random scenario (write values in random key order), {\tt FR}, simply writes key-value pairs
while db\_stress (data comparison with random write/delete/read), {\tt ST} shows complex 
operations including deletion.

For the data separation performance evaluation, 
we also implemented Autostream, manual technique, and
baseline which does not use the stream allocation policy.
We compared WAF of the existing techniques with {\sf PCStream}
for each scenario.




